\documentclass[10pt]{article}
\usepackage[utf8]{inputenc}
\usepackage[T1]{fontenc}
\usepackage{amsmath}
\usepackage{amsfonts}
\usepackage{amssymb}
\usepackage[version=4]{mhchem}
\usepackage{stmaryrd}
\usepackage{graphicx}
\usepackage[export]{adjustbox}
\graphicspath{ {./images/} }

\title{FOCS Answer key }

\author{}
\date{}


\begin{document}
\maketitle
Problem 1.1. The parity of an integer is 0 if it is even and 1 if it is odd. Which operations preserve parity:
(a) Multiplying by an even.
(b) Multiplying by an odd.
(c) Raising to a positive integer power.
answer
(a). Multiplying by an even number doesn't preserves parity.

Explanation:
Multiplying an even integer by an even number gives an even number like 4x2=8. But it fails to produce odd number if we multiply an odd number by an even number like 3x2=6 is not odd.
.........................

(b). Multiplying by an odd number preserves parity. Explanation: Multiplying any even integer by an odd number always gives an even number like 4x5=20.
Multiplying any odd integer by an odd number always gives an odd number like 3x5=15.

.........................
(c). Raising to a positive integer power preserves parity.

Explanation: Raising even number to any positive integer power always gives an even number like:
$2^1=2, 2^2=4, 2^5=32$
All of them are even.

Raising odd number to any positive integer power always gives an odd number like:
$3^1=3, 3^2=9, 3^5=243$
All of them are odd.
.........................

Hence correct choices are b and c .


Problem 1.2. What's wrong with this comparison: Google's nett worth in 2017, about $\$ 700$ billion, exceeds the GDP of many countries, e.g. Argentina's 2016-GDP was about $\$ 550$ billion. (Look up nett worth and GDP.)

Problem 1.3. Consider 2-contact EBOLA on a grid. You have one immunization vaccine. We show two different immunization scenarios, where you immunize the green square. Show the final infection in each case and determine which person you prefer to immunize? How many vaccines are needed to ensure that nobody else gets infected?

Problem 1.4. For the speed-dating problem with 16 people, $A, B, \ldots, P$ and four tables, arrange the rounds so that:
(a) In two rounds, everyone meets 6 people.
(c) In four rounds, everyone meets 12 people.
(b) In three rounds, everyone meets 9 people.
(d) In five rounds, everyone meets 15 people?

Problem 1.5 (Social Golfer Problem). 32 golfers form 8 groups of 4 each week. Each group plays a round of golf. No two golfers can be in the same group more than once. For how many weeks can this golfing activity go on?

(a) "Prove" that this golfing activity cannot go on for more than 10 weeks.

(b) Try to create a scheduling of players for as many weeks as you can. (10 is possible.)

(c) How is this problem related to the speed-dating problem?

In general you must schedule $g$ groups of golfers each of size $s$ for $w$ weeks so that no two golfers meet more than once in the same group. Given $(g, s, w)$, can it can be done and what is the schedule? This is a hard problem.

Problem 1.6. Students $A, \ldots, H$ form a friendship network (right). To advertise a new smartphone, you plan to give some students free samples. Here are two models for the spread of phone-adoption.

Model 1 (WEak Majority): People buy a phone if at least as many friends have the phone as don't. Model 2 (Strong Majority): People buy a phone if more friends have the phone than don't

(a) Use your intuition and determine the most "central" of the people in this friend-network.

(b) If you give a phone only to this central node, who ultimately has a phone in: (i) Model 1 (ii) Model 2?

(c) How many phones must you distribute, and to whom, so that everyone switches to your phone in Model 2?

(d) Repeat part (c), but now you cannot give a phone to the central node.

(A slight change to a model can have a drastic impact on the conclusions. A good model is important.)

Problem 1.7. Five radio stations (red stars) broadcast to different regions, as shown. The FCC assigns radio-frequencies to stations. Two radio stations with overlapping broadcast regions must use different radio-frequencies so that the common listners do not hear garbled nonsense. What is the minimum number of radio-frequencies the government needs?

Discrete math problems are like childhood puzzles. Parity, symmetry and invariance often yield simple solutions.
