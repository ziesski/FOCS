\documentclass[10pt]{article}
\usepackage[utf8]{inputenc}
\usepackage[T1]{fontenc}
\usepackage{amsmath}
\usepackage{amsfonts}
\usepackage{amssymb}
\usepackage[version=4]{mhchem}
\usepackage{stmaryrd}
\usepackage{graphicx}
\usepackage[export]{adjustbox}
\graphicspath{ {./images/} }

\title{FOCS Answer key }

\author{}
\date{}


\begin{document}
\maketitle
Problem 1.1. The parity of an integer is 0 if it is even and 1 if it is odd. Which operations preserve parity:
(a) Multiplying by an even.
(b) Multiplying by an odd.
(c) Raising to a positive integer power.
answer
(a). Multiplying by an even number doesn't preserves parity.

Explanation:
Multiplying an even integer by an even number gives an even number like 4x2=8. But it fails to produce odd number if we multiply an odd number by an even number like 3x2=6 is not odd.
.........................

(b). Multiplying by an odd number preserves parity. Explanation: Multiplying any even integer by an odd number always gives an even number like 4x5=20.
Multiplying any odd integer by an odd number always gives an odd number like 3x5=15.

.........................
(c). Raising to a positive integer power preserves parity.

Explanation: Raising even number to any positive integer power always gives an even number like:
$2^1=2, 2^2=4, 2^5=32$
All of them are even.

Raising odd number to any positive integer power always gives an odd number like:
$3^1=3, 3^2=9, 3^5=243$
All of them are odd.
.........................

Hence correct choices are b and c .


Problem 1.2. What's wrong with this comparison: Google's nett worth in 2017, about $\$ 700$ billion, exceeds the GDP of many countries, e.g. Argentina's 2016-GDP was about $\$ 550$ billion. (Look up nett worth and GDP.)

